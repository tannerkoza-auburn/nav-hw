% Copyright (C) Tanner Koza - All Rights Reserved
% Unauthorized copying of this file, via any medium is strictly prohibited
% Written by Tanner Koza <jtk0018@auburn.edu>, October 2022

% PROBLEM 4
\question
Consider the time-varying coordinate transformation matrix $C^n_b$ given below that describes the orientation of the body as it rotates with respect to the navigation frame.

\[C^n_b =
    \begin{bmatrix}
        cos(t)  & sin(t)sin(t^2)  & sin(t)cos(t^2) \\
        0       & cos(t^2)        & -sin(t^2)      \\
        -sin(t) & cos(t) sin(t^2) & cos(t)cos(t^2) \\
    \end{bmatrix} \]

\begin{parts}

    % PART A
    \part{Determine the analytic form of the time-derivative of $C^n_b$ (i.e. $\dot{C}^n_b = \frac{dC^n_b}{dt}$) via a term-by-term differentiation.}

    \solution
    The following derivative was determined using the \codeword{diff()} function in MATLAB.

    \begin{equation*}
        \begin{split}
            \dot{C_b^n} & =
            \begin{bmatrix}
                -\sin(t) & \sin(t^2)\cos(t)+2t\cos(t^2)\sin(t) & \cos(t^2)\cos(t)-2\sin(t^2)\,\sin(t) \\
                0        & -2t\sin(t^2)                        & -2t\cos(t^2)                         \\
                -\cos(t) & 2t\cos(t^2)\cos(t)-\sin(t^2)\sin(t) & -\cos(t^2)\sin(t)-2t\sin(t^2)\cos(t)
            \end{bmatrix}
        \end{split}
    \end{equation*}

    % PART B
    \part{Develop MATLAB functions which accept $t$ (i.e time) as a numerical input and return $C^n_b$ and $\dot{C}^n_b$, respectively, as numerical outputs.}

    \solution
    The functions \codeword{timeRotation()} and \codeword{timeRotationDot()} are appended to this document.

    % PART C
    \part{Using the $C^n_b$ and $\dot{C}^n_b$ functions from above, compute the angular velocity vector $\vec{\omega}^{\;n}_{nb}$ at time $t = 0$ seconds. (HINT: You might want to compute $\Omega^{n}_{nb}$)}

    \solution
    The skew-symmetric matrices in~\ref{part@4@3}-\ref{part@4@5} were calculated using the relationship described by Equation~\ref{eq:skew}. The corresponding unit vectors of instantaneous rotation were determined with the following:

    \begin{equation}
        \begin{split}
            \vec{k}^{\;n}_{nb} & =
            \frac{1}{2\sin(\theta)}
            \begin{bmatrix}
                \mathbf{C}^n_{b_{(3,2)}} - \mathbf{C}^n_{b_{(2,3)}} \\
                \mathbf{C}^n_{b_{(1,3)}} - \mathbf{C}^n_{b_{(3,1)}} \\
                \mathbf{C}^n_{b_{(2,1)}} - \mathbf{C}^n_{b_{(1,2)}} \\
            \end{bmatrix}
            \label{eq:unitvec}
        \end{split}
    \end{equation}

    $\theta$ in Equation~\ref{eq:unitvec} is determined by the following:

    \begin{equation}
        \begin{split}
            \theta & = \cos^{-1}\left(\frac{trace(\mathbf{C}^n_b) - 1}{2} \right)\\
        \end{split}
        \label{eq:theta}
    \end{equation}

    \begin{subparts}
        \subpart{What is the magnitude (i.e. $\dot{\theta}$, angular speed) of the angular velocity?}

        \solution
        Given,

        \[\Omega^{n}_{nb} =
            \begin{bmatrix}
                0  & 0 & 1 \\
                0  & 0 & 0 \\
                -1 & 0 & 0 \\
            \end{bmatrix}\]

        the angular velocity vector $\vec{\omega}^{\;n}_{nb}$ is the following:

        \[\vec{\omega}^{\;n}_{nb}=
            \begin{bmatrix}
                0 \\
                1 \\
                0 \\
            \end{bmatrix}\]

        Therefore, the magnitude can be calculated using \codeword{vecnorm()} in MATLAB as

        \[\dot{\theta} = 1~\unit{rad/s}\]

        \subpart{About what unit vector ($\vec{k}^{\;n}_{nb}$) has the instantaneous rotation occurred?}

        \solution
        Equation~\ref{eq:unitvec} is unable to yield a unit vector $\vec{k}^{\;n}_{nb}$ given $\theta =0$. This is because no rotation has occurred yet at $t = 0~\unit{s}$.
    \end{subparts}

    % PART D
    \part{Using the $C^n_b$ and $\dot{C}^n_b$ functions from above, compute the angular velocity vector $\vec{\omega}^{\;n}_{nb}$ at time $t = 0.5$ seconds.}

    \begin{subparts}
        \subpart{What is the magnitude (i.e. $\dot{\theta}$, angular speed) of the angular velocity?}

        \solution
        Given,

        \[\Omega^{n}_{nb} =
            \begin{bmatrix}
                0     & 0.48  & 1    \\
                -0.48 & 0     & 0.88 \\
                -1    & -0.88 & 0    \\
            \end{bmatrix}\]

        the angular velocity vector $\vec{\omega}^{\;n}_{nb}$ is the following:

        \[\vec{\omega}^{\;n}_{nb}=
            \begin{bmatrix}
                0.88  \\
                1     \\
                -0.48 \\
            \end{bmatrix}\]

        Therefore, the magnitude can be calculated using \codeword{vecnorm()} in MATLAB as

        \[\dot{\theta} = 1.41~\unit{rad/s}\]

        \subpart{About what unit vector ($\vec{k}^{\;n}_{nb}$) has the instantaneous rotation occurred?}

        \solution
        Equation~\ref{eq:unitvec} yields the following unit vector $\vec{k}^{\;n}_{nb}$ given $\theta = 0.5578~\unit{rad}$ from Equation~\ref{eq:theta}:

        \begin{equation*}
            \begin{split}
                \vec{k}^{\;n}_{nb} & =
                \begin{bmatrix}
                    0.439  \\
                    0.892  \\
                    -0.112 \\
                \end{bmatrix}
            \end{split}
        \end{equation*}

    \end{subparts}

    % PART E
    \part{Using the $C^n_b$ and $\dot{C}^n_b$ functions from above, compute the angular velocity vector $\vec{\omega}^{\;n}_{nb}$ at time $t = 1$ seconds.}

    \begin{subparts}
        \subpart{What is the magnitude (i.e. $\dot{\theta}$, angular speed) of the angular velocity?}

        \solution
        Given,

        \[\Omega^{n}_{nb} =
            \begin{bmatrix}
                0     & 1.68 & 1     \\
                -1.68 & 0    & -1.08 \\
                -1    & 1.08 & 0     \\
            \end{bmatrix}\]

        the angular velocity vector $\vec{\omega}^{\;n}_{nb}$ is the following:

        \[\vec{\omega}^{\;n}_{nb}=
            \begin{bmatrix}
                1.08  \\
                1     \\
                -1.68 \\
            \end{bmatrix}\]

        Therefore, the magnitude can be calculated using \codeword{vecnorm()} in MATLAB as

        \[\dot{\theta} = 2.24~\unit{rad/s}\]

        \subpart{About what unit vector ($\vec{k}^{\;n}_{nb}$) has the instantaneous rotation occurred?}

        \solution
        Equation~\ref{eq:unitvec} yields the following unit vector $\vec{k}^{\;n}_{nb}$ given $\theta = 1.3834~\unit{rad}$ from Equation~\ref{eq:theta}:

        \begin{equation*}
            \begin{split}
                \vec{k}^{\;n}_{nb} & =
                \begin{bmatrix}
                    0.660  \\
                    0.660  \\
                    -0.360 \\
                \end{bmatrix}
            \end{split}
        \end{equation*}

    \end{subparts}

    \part{In practice, direct measurement of the angular velocity vector $\vec{\omega}^{\;n}_{nb}$ can prove challenging, so a finite-difference approach may be taken given two sequential orientations represented by $C^n_b(t)$ and $C^n_b(t + \Delta t)$ a small time $\Delta t$ apart. Consider the approximate value of the angular velocity vector $\vec{\omega}^{\;n}_{nb}$ derived by using the finite difference
        \[\dot{C}^n_b(t) \approx \frac{C^n_b(t + \Delta t) - C^n_b(t)}{\Delta t} \]
        at times $t = 0$,$0.5$, and $1$ second. Compare the "analytic" values for $\dot{\theta}$ and $\vec{k}^{\;n}_{nb}$ (found in parts~\ref{part@4@2},~\ref{part@4@3}, and~\ref{part@4@4}) with your approximations from the finite difference using $\Delta t = 0.1$ seconds. How large are the errors?}

    \solution
    There are no errors between the unit vectors because they are calculated in the same manner. However, the angular velocity magnitudes are slightly different. The errors are depicted in Table~\ref{tbl:errors}.

    \begin{table}[h!]
        \centering
        \caption{Angular Velocity Errors}
        \def\arraystretch{1.5}
        \begin{tabular}{|c|c|c|c|}
            \hline
            Time (s)                         & $0.0$ & $0.5$    & $1.0$    \\
            \hline
            Analytical $\dot{\theta}$        & $1.0$ & $1.4142$ & $2.2361$ \\
            \hline
            Finite Difference $\dot{\theta}$ & $0.0$ & $1.3768$ & $2.2491$ \\
            \hline
            $error_{\dot{\theta}}$           & 1.0   & 0.0375   & -0.0130  \\
            \hline
        \end{tabular}
        \label{tbl:errors}
    \end{table}

    The errors between the analytical and finite difference angular velocities are on the order of $0.01~\unit{rad/s}$.

\end{parts}